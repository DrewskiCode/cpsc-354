
\documentclass{article}
\usepackage{amsmath}
\usepackage{amssymb}

\title{The MU-Puzzle}
\date{}

\begin{document}
\maketitle

MI $\rightarrow$ MU

\textbf{Rule 1:} If you possess a string whose last letter is \texttt{I}, add \texttt{U}.

\textbf{Rule 2:} Suppose you have \texttt{Mx}, you may add \texttt{Mxx}.

\textbf{Rule 3:} If \texttt{III} occurs in one of the strings, you may make a new string with \texttt{U} in place of \texttt{III}.

\textbf{Rule 4:} If \texttt{UU}, you can drop it.

\vspace{1em}

MI \\
MII $\; Mxx$ \\
MIIII $\; Mxx$ \\
MIIIIIIII $\; Mxx$ \\
MUIIU $\; MIU$ \\
$\varnothing$

\vspace{1em}

MI $\; \rightarrow$ use $Mxx$ rule $\infty$ times \\
MIIII... \\

I's are always a multiple of 2, thus not divisible by 3.

\vspace{1em}

\texttt{MUUU} \\
\texttt{MIII}

\vspace{1em}

\textbf{Rule 1} does not affect \# of I's. \\
\textbf{Rule 2} does not give divisible by 3. \\
\textbf{Rule 3} does not change the \# of I's. \\
\textbf{Rule 4} does not change the \# of I's. \\

Bottom line, you will never get rid of all of the I's.

\end{document}
