\documentclass{article}
\usepackage{amsmath}
\usepackage{amssymb}
\usepackage{hyperref} % For clickable table of contents

\title{CPSC 354 Report}
\author{Drew Floyd} % Replace "Student Name" with your actual name
\date{}

\begin{document}

\maketitle
\tableofcontents
\newpage

\section{The MU-Puzzle}

MI $\rightarrow$ MU

\textbf{Rule 1:} If you possess a string whose last letter is \texttt{I}, add \texttt{U}.

\textbf{Rule 2:} Suppose you have \texttt{Mx}, you may add \texttt{Mxx}.

\textbf{Rule 3:} If \texttt{III} occurs in one of the strings, you may make a new string with \texttt{U} in place of \texttt{III}.

\textbf{Rule 4:} If \texttt{UU}, you can drop it.

\vspace{1em}

MI \\
MII $\; Mxx$ \\
MIIII $\; Mxx$ \\
MIIIIIIII $\; Mxx$ \\
MUIIU $\; MIU$ \\
$\varnothing$

\vspace{1em}

MI $\; \rightarrow$ use $Mxx$ rule $\infty$ times \\
MIIII... \\

No matter what Rule you use you will never be able to get 0 Mod3, because I will always be 1 mod 3 or 2 mod 3

\vspace{1em}

\texttt{MUUU} \\
\texttt{MIII}

\vspace{1em}

\textbf{Rule 1} does not affect \# of I's. \\
\textbf{Rule 2} does not give 0 mod 3. \\
\textbf{Rule 3} does not solve the problem as removing 3 I's does not change the output of mod3. \\
\textbf{Rule 4} does not change the \# of I's. \\

We can never get rid of all of the I's, 0 mod 3 is not possible. Thus you cannot get MU from MI.

\end{document}
